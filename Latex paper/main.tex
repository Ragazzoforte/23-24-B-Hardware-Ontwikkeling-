\documentclass{article}

% Useful packages
\usepackage{multicol}
\setlength{\columnsep}{1cm}

% Set page size and margins
% Replace `letterpaper' with `a4paper' for UK/EU standard size
\usepackage[a4paper,top=2cm,bottom=2cm,left=2cm,right=2cm,marginparwidth=1.75cm]{geometry}

\usepackage{caption}
\usepackage{amsmath}
\usepackage{siunitx}
\usepackage{wrapfig}
\usepackage{float}
\usepackage{graphicx}
\graphicspath{{images/}}
\usepackage{subcaption}
\usepackage[colorlinks=true, allcolors=blue]{hyperref}
\usepackage{xcolor}
\usepackage{listings}
\usepackage{import}

% Language setting
% Replace `english' with e.g. `spanish' to change the document language
\usepackage[dutch]{babel}

\title{HARONT DCDC Converter}

\author{
  Vollmuller, Michel\\
  1809572\\
  \texttt{michel.vollmuller@student.hu.nl}
  \and
  Hemstra, Jelmer\\
  1810225\\
  \texttt{jelmer.hemstra@student.hu.nl}
}

\begin{document}
\maketitle

\begin{abstract}
    Hier komt een abstract.
\end{abstract}

% hoofdstuk 1
\section{Systeem uitleg}

hier komt de uitleg van het systeem.

% hoofdstuk 2
\section{DC Gedrag}

DC gedrag
Hoe reageert het systeem op veranderingen. Hoe stabiel is systeem; efficiëntie tov load

% hoofdstuk 3
\section{AC Gedrag}

AC gedrag
Hoe groot is de output storing ten opzichte van de load

% hoofdstuk 4
\section{Startgedrag}

Startgedrag
Startgedrag met verschillende loads en verklaring


\section{Duiding en Conclusie}
Duiding en conclusie.
Het systeem is goed gelukt, de PI regelaar werkt goed en het doen van instelbaarheid tussen 0 en 7.5 volt is behaalt. 
Maar helaas is het doen van 50 mA belasting op 7.5v niet gelukt. 
De maximale belasting op 7.5v is 15 mA. 

\section{Aanbeveling}
Als we het project over zouden doen zouden we een hoop anders doen. 
Natuurlijk is de belangrijkste een andere mosfet gebruiken. 

%Bibliography
\bibliographystyle{IEEEtran}
\bibliography{bib}
\appendix

\end{document}